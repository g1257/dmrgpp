\documentclass{article}
\usepackage{marginnote}

\begin{document}

\title{DMRG++ Feature Proposal:\\
Tracking Unnatural Operators}
\author{various}
\maketitle

\marginnote{Definition of Natural}{\sc An Operator is natural} or
more precisely \emph{strictly DMRG-natural}
if it appears in a Hamiltonian connection. In the Hubbard model
with Hamiltonian
\begin{equation}
H=\sum_{\sigma, i\neq j}t_{i,j} c^\dagger_i c_j + U\sum_i n_i,
\end{equation}
the operators $c_i$ are natural (and $c^\dagger_i$) but not $n_i$.
The naturality of an operator is thus a model-dependent property.
For example, $n_i$ isn't natural in the Hubbard model as we've just
seen, but it is natural
in the Hubbard extended model or in the t-J model, due to the $n_i n_j$ connection.
\emph{Unnatural operators} are those which are not necessarily natural.
 The member function \texttt{naturalOperator}
of each Model provides all natural operators and a few others
in the one-site Hilbert space.

\marginnote{Definition of Resilient}
Let $A$ be an one-site operator and $\{B_0, B_1, \cdots B_{n-1}\}$ be a set
of $n$ one-site operators, such that there exists some function $f$ such that
\begin{equation}
A = f(B_0, B_1, \cdots, B_{n-1}).
\label{eq:AfB}
\end{equation}
Let $\{W_0, W_1, \cdots, W_{m-1}\}$ be a set of $m$ \emph{sequential} DMRG 
 transformations and let 
 $\tilde{X} = 
 W^\dagger_{m-1} W^\dagger_{m-2}\cdots W^\dagger_0 X W_0\cdots W_{m-2} W_{m-1}$ be
 the DMRG transformed operator of $X$ after these $m$ transformations.
Eq.~(\ref{eq:AfB}) is said to be \emph{DMRG-transformation resilient}
or just DMRG-resilient, if
$\tilde{A} =  f(\tilde{B}_0, \tilde{B}_1, ...)$ for any sequential set.
For example, $n_\uparrow=c^\dagger_\uparrow c_\uparrow$ is \emph{not} DMRG-resilient,
but  $c^\dagger_\uparrow = (c_\uparrow)^\dagger$ and
$n=n_\uparrow + n_\downarrow$ both are.

\marginnote{Minimally complete natural set}
A set of operators $\{B_0, B_1, \cdots, B_{n-1}\}$ is a \emph{minimally complete
natural set}, if (i) all $B_i$ are natural, (ii) there is no DMRG-resilient
equation connecting one of them to the others, and (iii) if $A$ is a 
natural operator, either $A=B_i$ for some $i$ or there exists $f$ such that
$A = f(B_0, B_1, \cdots, B_{n-1})$ is resilient.

\marginnote{Tracking an operator}{\sc Tracking an operator}
$A_i$ means to (i) create it of rows and columns equal to the 
one-site Hilbert space basis when
site $i$ is first in view. (ii) To transform it by DMRG transform $W$ as
$\tilde{A}_i = W^\dagger A_i W$, and to keep transforming it in this way
as the lattice is swept. DMRG++ must track all operators
of a minimally complete
natural set in order to build the Hamiltonian. FIXME: EXPLAIN WHY.

The targeting abstract module defines an interface to the
density matrix module that specifies what vectors to target, with what
weights, and how to construct and update said vectors.
\marginnote{Definition of targeting and NGSTs}
There are different ways to implement this abstract interface, and each
way defines a \emph{instance} targeting. For instance, \texttt{GroundStateTargeting}
is the simplest of them as it targets \emph{only} the ground state vector.
All other instances are referred to as \emph{non ground state targets} or NGSTs; 
see \texttt{EngineTargeting*.h} in DMRG++ for a list. For example, 
\texttt{TargetingTimeStep.h}
does t{\sc{DMRG.}}

\marginnote{Rationale} To obtain correlations $\langle gs| A_i B_j |gs\rangle$
\emph{in-situ} we need to target vectors $B_j |gs\rangle$.
This approach has two problems: (i) We cannot use more than one \emph{NGSTs}
as \texttt{TargetingInSitu} is already in use, and
even if we could (ii) the vectors to target might become too many if 
targeting other things---considering
that we have only one MPS. Tracking Operators $A_i$ and $B_j$ would make
the braket $\langle gs| A_i B_j |gs\rangle$ directly available, and could
be used in \emph{NGSTs} because this feature would be independent
of the targeting feature, allowing for the calculation of
 $\langle P0| A_i B_j |gs\rangle$ or for other bra and kets.

\marginnote{Syntax and User Perspective} 
We would run DMRG++ with\footnote{Instead of the bra and ket gs, other
non ground state vectors may be used, like P0, P1, etc, depending on the
NGST in use. Note that there are two ways of specifying operators in
the command line: (1) by filename, as in ``:filename'' where \$ substitution
is applied, (2) by natural operator \texttt{\_name[site]?dof'}, where
site, dof and the transpose single quote are all optional. 
Before this feature the underscore did not exist, and the operator name
had to be natural. This feature
adds an optional underscore \_ that allows for an unnatural operator.}

\begin{verbatim}
./dmrg -f input.inp '<gs|_A;_B|gs>'
\end{verbatim}
where $\_A$ and $\_B$ are unnatural operators that are specified to be
tracked in the input file as follows.
\begin{verbatim}
TrackUnnaturalOperatorsTotal=3
TrackUnnaturalOperatorsDelay 3    l0 l1 l2
<block0>
<block1>
<block2>
\end{verbatim}
where we here wish to track 3 unnatural operators. 
The delay specifies the delay of each for its creation, in units
of finite loops.
Each \texttt{<block>} has the following syntax.
\begin{verbatim}
TrackUnnaturalOperatorName=_A
<file form operator specification>,
\end{verbatim}
Note that the name of the to-be-tracked operators starts (or ends haven't
decided yet FIXME) with a specific prefix which shall be underscore \_
(or something else haven't
decided yet FIXME)/
Finally, the block \texttt{<file form operator specification>} is well-known
and has been described elsewhere.

\marginnote{Implementation and Coder Perspective}To be Written TBW.
\end{document}